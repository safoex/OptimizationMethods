\documentclass[a4paper,11pt]{scrartcl} 
\usepackage [ warn ]{ mathtext}
\usepackage[top=10mm]{geometry}
\usepackage[T1]{fontenc}
\usepackage[utf8x]{inputenc} %кодировка 
\usepackage{mathrsfs}
\usepackage[english,russian]{babel}
\usepackage{indentfirst}
\usepackage{mathtools}
\usepackage{amssymb}
\usepackage{amsfonts}
\usepackage{braket}
\newcommand{\pd}{\partial}

\title{Problem 1}
\author{Team 2}
\date{January 25, 2017}
\begin{document}
\maketitle
Newton's equations:
\begin{equation}
\left(
\begin{array}{c}
 m \left(v_x' t\right) \\
 m \left(v_y' t\right) \\
\end{array}
\right)=\left(
\begin{array}{c}
 m f(t) \cos  \alpha  (t)   -k \left(v_x t\right) \\
 -\left(v_y t\right) k-g m+m  f(t)  \sin  \alpha ( t) \\
\end{array}
\right)
\end{equation}
Here $f(t) = {F(t) \over m}$\newline
Force $f(t)$ acting under angle $\alpha(t)$ 
Let's solve (1) for $v_x, v_y$
\begin{equation}
v_x = 
e^{-\frac{k t}{m}} \int_0^t  f(t)  e^{\frac{k t_2}{m}} (\cos  (\alpha(t_2) )) \, dt_2
\end{equation}
\begin{equation}
v_y = e^{-\frac{k t}{m}} \int_0^t e^{\frac{k t_2}{m}} ( f(t)  (\sin  (\alpha(t_2)))-g) \, dt_2
\end{equation}
\begin{equation}
\left(a_y+g+\beta  v_y\right){}^2+\left(a_x+\beta  v_x\right){}^2= f^2(t) 
\end{equation}
From this we can derive $a_x(t)$ and find $v_x(t)$
\begin{equation}
v_x(t) = e^{- \beta t} \int_0^t e^{\beta  t_2} \sqrt{f(t_2)^2-\left(g+v'_y(t_2)+\beta  v_y(t_2)\right)^2} \, dt_2
\end{equation}
\linebreak
\begin{equation}
L_x = L =\int_0^{t_1} e^{- \beta t} \int_0^t e^{\beta  t_2} \sqrt{f(t_2)^2-\left(g+\beta   v_y\left(t_2\right)+ v'_y\left(t_2\right)\right){}^2} \, dt_2 \, dt
\end{equation}
\begin{equation}
L_y = \int_0^{t_1} \left(\int_0^t  v_y'(t_2) \, dt_2\right) \, dt=0
\end{equation}
\linebreak
\begin{equation}
F(t,t_2,v_y,v'_y) = e^{\beta  (t-t_2)} \sqrt{f(t)^2-(g+\beta  v_y+v'_y)^2}+\lambda  v'_y
\end{equation}
\begin{equation}
\frac{\partial F}{\partial v_y} - \frac{\partial}{\partial t}\frac{\partial F}{\partial v'_y} = 0
\end{equation}
\begin{equation}
f'\left(t \right) \left(g+\beta   v_y\left(t\right)+ v'_y\left( t\right)\right)= f(t)  \left(\beta  v'_y\left(t \right)+ v''_y\left( t\right)\right)
\end{equation}
From (10) we can see that force should act on a constant angle \linebreak 

For the case $F(t) = F_0 (1 + {t \over t_0})^2$ and $t_0 >> {m \over k} $
\begin{equation}
v_x(t) = C_2 e^{- \beta t} \left(\beta  \text{Ei}((t+t_0) \beta )-\frac{e^{\beta  (t+t_0)}}{t+t_0}\right)+C_1 e^{- \beta t}-\frac{g}{\beta }
\end{equation}
\begin{equation}
v_x(t) = \frac{f_0   \cos \alpha}{\beta  \left(\frac{t}{t_0}+1\right)^2}
\end{equation}
\begin{equation}
v_y(t) = \frac{\frac{f_0  \sin \alpha }{\left(\frac{t}{t_0}+1\right)^2}-g}{\beta }
\end{equation}
\begin{equation}
\frac{g (t_0+t_1)-f_0 t_0   \sin \alpha}{\beta }=0
\end{equation}
\begin{equation}
t_1 = \frac{f_0 t_0  \sin \alpha - g t_0}{g}
\end{equation}
Finally, 
\begin{equation}
L = \frac{\sqrt[3]{f_0} t_0 \left(f_0^{2/3}-g^{2/3}\right) \sqrt{1-\frac{g^{2/3}}{f_0^{2/3}}}}{\beta } 
\end{equation}

\end{document}